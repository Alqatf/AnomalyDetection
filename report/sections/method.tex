\section{Methodology}
\subsection{System overview}
\begin{enumerate}
\item Affinity matrix
\item Clustering
\item Outlier detection
\end{enumerate}

\subsection{Definition of anomaly}
I therefore chose the unsupervised definition of anomalies:
\begin{itemize}
\item frequently occurred patterns are normal
\item the pattern dissimiliar to the majority of normal patterns in anomalous.
\end{itemize}

\subsection{Step 1. Affinity matrix}
\subsubsection{Data pre-processing}
\begin{itemize}
\item categorical value to integer e.g.) service-type (ftp-data,http,etc).
\item log of big number e.g.) duration, src-bytes.
\end{itemize}

\subsubsection{Learning similarity}
EM algorithms to learn similarity. Learned 234(=2 x 3 x 39) Gaussian mixture model in total.
\begin{itemize}
\item 2 : one for normal, one for abnormal.
\item 3 : each per each protocol e.g.) udp, icmp, tcp.
\item 39 : for all attributes.
\end{itemize}

\subsubsection{Measuring similarity}
Gaussian mixture models from training set helps to measure those two similarity per each connection.
\begin{itemize}
\item Similarity to known normal behavior.
\item Similarity to known abnormal behavior.
\end{itemize}

\subsubsection{Affinity matrix computation}
The data points are associated with each other by pairwise similarity.
\begin{itemize}
\item Construct similarity matrix with distance metric.
\item Construct affinity matrix from similarity matrix with 8-nearest neighbors algorithm.
\end{itemize}

\subsection{Step 2. Clustering}
\subsubsection{Number of clusters prediction}
Predict number of clusters based on the eigengap.
\subsubsection{Spectral clustering algorithm}
Normalized cut algorithm. \cite{jianbo00}
\subsubsection{Representative of clusters}
Clusters are represented by the mean and variance.

\subsection{Step 3. Detecting anomaly from clusters}
Distance based outlier detection is used.
\begin{itemize}
\item It do not require any prior knowledge.
\item k-nearest neighbor outlier detection algorithm. \cite{knorr00}
\end{itemize}

