\section{Connection Similarity}
The problem of comparing network connection similarity has been an important problem. \cite{}
After reviewing the families of proposed schemes, I indentified similarity and density approaches as the promising for the problem. 
In Section~\ref{subsec:normalabnormalsimilarity}, I propose new approaches to measure similarity.
In Section~\ref{subsec:densitysimilarity}, I describe density similarity measurement which is relied on the way of representatives of the clusters and threshold.

\subsection{Problem formulation}
\label{subsec:problemformulation}
Given dataset, we can estimate normal/abnormal mixture models, and density of normal connections from training set.
Anomalies can be detected by comparing clusters which is computed by similarity score, or by checking density against threshold.

%\subsection{Requirement}

\subsection{Normal/Abnormal Similarity}
\label{subsec:normalabnormalsimilarity}
Two nodes are similar if their similarity to normal/abnormal mixture models are similar.
Mixture models can be learned by EM algorithm, and compute the similarity score, log probability of each connections, under the model.
The EM approach is guaranteed to converge to a local optimum on a given input.
We define the similarity between two nodes using cosine distance of normal and abnormal similarity scores.

\begin{equation}
    sim(V, V') = \frac{A \cdot B}{|A| |B|}
\end{equation}

\subsection{Density Similarity}
\label{subsec:densitysimilarity}
A cluster is normal if its mean and variance to normal gaussian model is similar. 
We can adjust threshold if false-positive or false-negative is high. 
