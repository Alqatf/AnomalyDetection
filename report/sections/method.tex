\section{Network Connection Similarity}
\label{sec:connectionsimilarity}
A network connection is a connection between computers in the Internet to transmit and receive data. 
Connection data created by host computers is important to monitor of a network status.
Network connection similarity helps to measure the similarity between network connections, and is essential to construct clusters. 
%, and requires comparison methods that helps to measure the similarity between network connections. 
%Network connection similairy methods provide how the data points are similar each other and is essential to construct clusters. 
After reviewing the families of proposed schemes, I indentified similarity and density approaches as a promising approach to solve the problem.
%In this report, we are gonig to measure similarity between network connections. 
%Since there is not only one normal state, I generated mixture models for normal connections for each protocols and attributes.
%so the problem of comparing similarity has been an important problem. %\cite{}
\newline
In Section~\ref{subsec:problemformulation}, I describe the problem.\newline
In Section~\ref{subsec:normalabnormalsimilarity}, I propose new approaches to train similarity cost function.\newline
In Section~\ref{subsec:densitysimilarity}, I describe density similarity measurement which is relied on the way of representatives of the clusters and threshold.\newline
%In Section~\ref{subsec:learningsimilarity}, I describe how the algorithm learn mixture models with training set.\newline
\subsection{Problem Formulation}
\label{subsec:problemformulation}
Given dataset about network connection data, we can learn normal and abnormal mixture models in order to measure similarity score, and estimate a density of normal connections from training set.
Anomalies can be detected by cluster algorithm based on affinity matrix which is computed by similarity score, or by comparing density of clusters against threshold.

\subsection{Normal and Abnormal Network Connection Similarity}
\label{subsec:normalabnormalsimilarity}
Two nodes are similar if their similarity to normal and abnormal mixture models are similar. 
Mixture models can be learned by EM algorithm, and it is guaranteed to converge to a local optimum on a given input. 
They compute the similarity score, log probability of each connections, under the model. 
In order to define the similarity between two nodes, I found cosine distance of normal and abnormal similarity scores shows better performance. 
As a result I have mixture modesl seperately for 23 known classes, three protocols and 39 attributes. 
Each attributes have different correlation to the result \cite{olusola10}\cite{kayacik05}, so I give different weights on them.
%Learned $234(=2 \times 3 \times 39)$ Gaussian mixture models in total.
%\begin{itemize}
%\item 2 : one for normal, one for abnormal.
%\item 3 : each per each protocol e.g.) udp, icmp, tcp.
%\item 39 : for all attributes.
%Since the data fit the gaussian and a sufficient number of data points are available to learn the parameters of the model, the model can be learned.
%\begin{equation}
%    sim(V, V') = \frac{A \cdot B}{|A| |B|}
%\end{equation}
%

\subsection{Connection Density Similarity}
\label{subsec:densitysimilarity}
A cluster is abnormal if its density differs from density of known normal connections even though the cluster is similar to normal behaviour \cite{ester96}. 
For anomaly detection, only density of each cluster is compared with the density of known normal connections. 
If the density of a cluster is higher than a supposed density in the region, the cluster is classified as an anomaly. 
So it allows detecting unknown anomalies which is similar to known normal connections. 
In contrast to the known normal and abnormal connection similarity, it does not need to use known anomalies that appear in the training set. 
%This is illustrated in (Figure) where one cluster over the distribution and threshold. 
We can also define threshold and adjust threshold if false-positive or false-negative is high. 

%\begin{equation}
%    d = \sum_X \sum_Y 
%\end{equation}
%\subsubsection{Measuring similarity}
%\label{subsec:densitysimilarity}
%Gaussian mixture models from training set helps to measure those two similarity per each connection.
%\begin{itemize}
%\item Similarity to known normal behavior.
%\item Similarity to known abnormal behavior.
%\end{itemize}

