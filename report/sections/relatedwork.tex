\section{Related works}
% What is different approaches/alternative to NIDS has there been before?
Semi-supervised or unsupervised approaches are preferred in intrusion detection\cite{chandola09} because it can handle imbalanced datasets.
Clustering analysis is an important tool for those approaches so also been studied in context of network anomaly detection. 
These approaches to network intrusion detection systems are mostly either distance-based\cite{knorr00}\cite{ramaswamy00} or density-based methods\cite{breuning00}{aggarwal01}. 

% In what way they differ from each other
However distance-based or density-based methods are hard to be easily generalized in multi-dimensional data because the neighborhood can be arbitrary. 
A major drawback to clustering algorithm is that anomalies which should not belong to any cluster can be assigned to a larger cluster. 
% because clustering algorithm force every instance to be assigned to a cluster. 
Also it is subjective to choose a number of clusters or neighborhood in advance, so not applicable to the case of dynamic intrusion detection. 

One of the strategies in order to solve intractable the optimization problem is a eigenvalue decomposition and graph cut algorithm. 
However the naive graph cut algorithm is NP-hard problem and noise-sensitive. 
To diminish those weaknesses, normalized cut\cite{jianbo00} is proposed. 
It solves NP-hard problem by approximate solution.
Spectral clustering according to Jordan and Francis \cite{jordan04} and Ng, Jordan and Weiss \cite{ng01} proposed a new algorithm for spectral clustering with objective function with different normalized terms that minimize the error.
Dhillon et al. \cite{dhillon04}, and Ding and He \cite{cding04} showed objective function is equivalent to weighted kernel k-means algorithm.

% In what way they differ from what we do here
The advantages of spectral clustering is it only requires pairwise similarity of data points. 
However mostly it is used in the spatial analysis such as image segmentation. 
In this report, I use spectral clustering algorithm for intrusion detection. 
%I uses histogram and mixture models to measure its similarity.

% Why is our approach better than what has done before. does it solve a partly new aspect of the problem or does it simply perform better
I use NSL-KDD99 Dataset for the report. 
It is mainly used in this subject\cite{tavallaee09} and its relevance of each feature in the data set is also studied\cite{olusola10} \cite{kayacik05}. 

