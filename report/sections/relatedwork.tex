\section{Related works}
% What is different approaches/alternative to NIDS has there been before?
Semi-supervised or unsupervised approaches are preferred in Anomaly Detection\cite{chandola09} because labeled data corresponding to normal behaviour is available while labels for intrusions are not. 
Different approaches to network intrusion detection systems, such as \begin{inparaenum}[\itshape a\upshape)] \item statistical profiling using histogram, \item mixture of models, \item neural networks, \item support vector machines, and \item rule-based system\end{inparaenum}. NSL-KDD99 Dataset is mainly used in this subject\cite{tavallaee09}. 
The relevance of each feature in the data set is also studied\cite{olusola10} \cite{kayacik05}. 

% In what way they differ from each other
The rule-learning engine RIPPER and ToolDiag is applied to the training data with feature labeled as normal or intrusion and generate rules for classification. 
Minnesota Intrusion Detection System selected LOF detection method. 
Also fuzzy based methods are proposed.

%% In what way they differ from what we do here
%I assumes the data 
%to decide k value.
%I uses histogram and mixture models to measure its similarity.

% Why is our approach better than what has done before. does it solve a partly new aspect of the problem or does it simply perform better
In this report, I use spectral clustering algorithm for intrusion detection. 
The advantages of spectral clustering is it only requires pairwise similarity of data points. 

\begin{table}[h]
%\begin{tabular}{|r|l|}
%\end{tabular}
(table)
\caption{Intrusion detection approaches}
\label{fig:refSingleRobot1}
\end{table}

