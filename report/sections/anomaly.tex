\section{Potential anomalies}
A potential anomaly occurs when data point is \begin{inparaenum}[\itshape a\upshape)]
\item known anomalies which is similar to known abnormalies, %dissimilar to normal connections,
\item unknown anomalies which is similar to known abnormalies, %or dissimilar to normal connections,
\item or unknown anomalies which is similar to known normal behaviours but with very higher density.
\end{inparaenum} 

Since a goal of this paper is an intrusion detection, I will give examples of this types of anomalies from the NSL-KDD dataset \cite{tavallaee09}. 
The dataset is for building an intrusion detection task to detect network anomalies which is possible network attacks. 
It contains $21$ known anomalies, and $13$ unknown anomalies. 
It also contains four more unknown anomalies, but they are similar to normal data points.
A detail of the dataset is specified in~\ref{subsec:datasetandsetup}. 

%%Point anomaly
%%Collective anomaly
%\begin{figure}[htb2]
%\begin{center}
%(similar to normal)
%(similar to abnormal)
%(similar to normal, but very high density)
%%\includegraphics[width=2.5in,angle=0]{}
%\end{center}
%\caption{Examples of anomalies that are potential anomalies}
%\label{fig:refSingleRobot1}
%\end{figure}
\begin{table}[h]
\begin{center}
\begin{tabular}{| l | l | p{5cm} |}
\hline
Type of Anomaly & Similar to & Classes in NSL-KDD \\
\hline
Known anomalies & Known anomalies & guess-passwd, spy, ftp-write, nmap, back, multihop, rootkit, pod, portsweep, perl, ipsweep, teardrop, satan, loadmodule, buffer-overflow, phf, warezmaster, imap, warezclient, land, neptune, smurf \\ %21
\hline
Unknown anomalies & Known anomalies & processtable, named, udpstorm, sqlattack, ps, httptunnel, apache2, saint, mscan, xterm, worm, xlock, xsnoop \\ %13
\hline
Unknown anomalies & Normal behaviours & sendmail, mailbomb, snmpguess, snmpattack \\ % 4
\hline
\end{tabular}
\end{center}
\caption{Abnomal classes in the NSL-KDD99}
\label{fig:anomalyclasses}
\end{table}
