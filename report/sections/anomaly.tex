\section{Potential anomalies}

\begin{figure}[htb2]
\begin{center}
(figure)
%\includegraphics[width=2.5in,angle=0]{}
\end{center}
\caption{Examples of anomalies that are potential anomalies}
\label{fig:refSingleRobot1}
\end{figure}

%\subsection{Definition of anomaly}
An anomaly occurs when the data does not \begin{inparaenum}[\itshape a\upshape)]
\item similar to known abnormal connections,
\item dissimilar to known normal connections,
\item or similar to known normal connections but with very higher density.
\end{inparaenum}

With this in mind, I select the unsupervised definition of anomalies:
First, the pattern dissimiliar to the majority of normal patterns is anomalous.
Next, infrequently occurred density are normal


