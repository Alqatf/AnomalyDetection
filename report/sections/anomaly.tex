\section{Potential anomalies}
A potential anomaly occurs when data point is not \begin{inparaenum}[\itshape a\upshape)]
\item similar to known abnormal connections,
\item dissimilar to known normal connections,
\item or similar to known normal connections but with very higher density.
\end{inparaenum} 





%Point anomaly
%Collective anomaly
\begin{figure}[htb2]
\begin{center}
(similar to normal)
(similar to abnormal)
(similar to normal, but very high density)
%\includegraphics[width=2.5in,angle=0]{}
\end{center}
\caption{Examples of anomalies that are potential anomalies}
\label{fig:refSingleRobot1}
\end{figure}

\begin{table}[h]
\begin{center}
\begin{tabular}{| l | l | p{5cm} |}
\hline
Type of Anomaly & Similar to & classes \\
\hline
Known anomalies & Abnormal model & guess-passwd, spy, ftp-write, nmap, back, multihop, rootkit, pod, portsweep, perl, ipsweep, teardrop, satan, loadmodule, buffer-overflow, phf, warezmaster, imap, warezclient, land, neptune, smurf \\ %21
\hline
Unknown anomalies & Abnormal model & processtable, named, udpstorm, sqlattack, ps, httptunnel, apache2, saint, mscan, xterm, worm, xlock, xsnoop \\ %13
\hline
Unknown anomalies & Normal model & sendmail, mailbomb, snmpguess, snmpattack \\ % 4
\hline
\end{tabular}
\end{center}
\caption{Abnomal classes in NSL-KDD99}
\label{fig:refSingleRobot1}
\end{table}
