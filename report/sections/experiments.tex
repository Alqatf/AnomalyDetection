\section{Experiments}
I performed experiments to evaluate the performances of similarity measures and algorithms in Section~\ref{sec:connectionsimilarity}.
\newline
In Section~\ref{subsec:datasetandsetup}, I describe the dataset that I used for the experiements.\newline
In Section~\ref{subsec:detectinganomalies}, I evaluate how successful the algorithms are in detecting different types of anomalies.

\subsection{Dataset and Setup}
\label{subsec:datasetandsetup}
For experiments, I select KDD Data set. 
KDDCUP'99 data set is widely used for network-based anomaly detection. 
New version of KDD data set, NSL-KDD which has advantages over the original data set, is used in this report as an effective benchmark.

\begin{figure}[htb2]
\begin{center}
%\item Selection of raw data
%\begin{inparaenum}[\itshape a\upshape)]
%\item Data preprocessing
%\item Data transformation
%\item Affinity matrix computation
%\item Clustering
%\item Outlier detection
%\end{inparaenum}
\end{center}
\caption{Overview of Intrusion Detection System}
\label{fig:refSingleRobot1}
\end{figure}

\begin{figure}[htb2]
\begin{center}
\end{center}
\caption{Similarity of normal and abnormal connections in training set} % I may show rest of data in appendix
\label{fig:refSingleRobot1}
\end{figure}

%\subsection{Computing affinity matrix}
%%The processing steps of the approach can be summerized as follows:
%%1) Training mixture model with training set containing records of both normal and anomalous connections.
%%2) The data are divided into different clusters for normal and anomalous connections using Spectral clustering algorithm.
%\subsubsection{Data pre-processing}
%\begin{itemize}
%\item categorical value to integer e.g.) service-type (ftp-data,http,etc).
%\item log of big number e.g.) duration, src-bytes.
%\end{itemize}
%\subsubsection{Affinity matrix computation}
%The data points are associated with each other by pairwise similarity.
%\begin{itemize}
%\item Construct similarity matrix with distance metric.
%\item Construct affinity matrix from similarity matrix with 8-nearest neighbors algorithm.
%\end{itemize}
%\subsection{Clustering}
%%\subsubsection{Number of clusters prediction}
%%Predict number of clusters based on the eigengap.
%%\subsubsection{Spectral clustering algorithm}
%%Normalized cut algorithm. \cite{jianbo00}
%%\subsubsection{Representative of clusters}
%%Clusters are represented by the mean and variance.
%\subsection{Detecting anomaly from clusters}
%%Distance based outlier detection is used.
%%\begin{itemize}
%%\item It do not require any prior knowledge.
%%\item k-nearest neighbor outlier detection algorithm. \cite{knorr00}
%%\end{itemize}

\subsection{Detecting Anomalies}
\label{subsec:detectinganomalies}
In this section, I describe how the similarity measures can be used to detect point and collective anomalies.

\subsubsection{Known Anomalies}
We see that normal and abnormal connection similarity are sensitive to known anomaly. The similarity coputed by mixture models gives similarity scores that are used to find k nearest neighborhood.
\begin{figure}[htb2]
\begin{center}
\end{center}
\caption{Similarity of normal and abnormal connections in training set} % I may show rest of data in appendix
\label{fig:refSingleRobot1}
\end{figure}

\subsubsection{Unknown Anomalies}
We see that normal and abnormal connection similarity are also sensitive to most of unknown anomaly. However, we have four classes that is similar to normal connections. Connection density similarity measure is sensitive in this type of anomaly if it is above its threshold. However this approach does not work if the data points are insufficient to form high density.

\begin{figure}[htb2]
\begin{center}
\end{center}
\caption{Similarity to normal and abnormal connections in test set} % I may show rest of data in appendix
\label{fig:refSingleRobot1}
\end{figure}

