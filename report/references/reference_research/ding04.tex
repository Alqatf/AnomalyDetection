\section{Linearize Cluster Assignment via Spectral Ordering}

\textit{Linearize Cluster Assignment via Spectral Ordering} by Chris Ding. \\
Cited by 69. \textit{Machine learning. ACM, 2004.}
\newline

\textbf{Main point} is that \begin{inparaenum}[\itshape a\upshape)]
\item that K-way clustering method depends on a linear ordering provided by the spectral ordering, 
\item paper provide a ordering objective function.
\end{inparaenum}

\subsection{Linearized cluster assignment}
The linearized assignment algorithm depends on three techniques \begin{inparaenum}[\itshape a\upshape)]
\item an ordering of the data objects, 
\item clustering crossing,
\item the connectivity matrix,
\end{inparaenum}

The actual linearization is performed via the cluster clustering, the sum of similarities symmetrically across a cut point along the linear ordering.

